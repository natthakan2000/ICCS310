% You should title the file with a .tex extension (hw1.tex, for example)
\documentclass[a4paper, 11pt]{article}

\usepackage{amsmath}
\usepackage{amssymb}
\usepackage{fancyhdr}

\usepackage[margin=1in]{geometry}

\newcommand{\question}[2] {\vspace{.25in} \hrule\vspace{0.5em}
\noindent{\bf #1: #2} \vspace{0.5em}
\hrule \vspace{.10in}}
\renewcommand{\part}[1] {\vspace{.10in} {\bf (#1)}}

\newcommand{\myname}{Natthakan Euaumpon}
\newcommand{\myemail}{natthakaneuaumpon@gmail.com}
\newcommand{\myhwnum}{3}

\setlength{\parindent}{0pt}
\setlength{\parskip}{5pt plus 1pt}
 
\pagestyle{fancyplain}
\lhead{\fancyplain{}{\textbf{HW\myhwnum}}}      % Note the different brackets!
\rhead{\fancyplain{}{\myname\\ \myemail}}
\chead{\fancyplain{}{ICCS310}}

\begin{document}

\medskip                        % Skip a "medium" amount of space
                                % (latex determines what medium is)
                                % Also try: \bigskip, \littleskip

\thispagestyle{plain}
\begin{center}                  % Center the following lines
{\Large ICCS310: Assignment \myhwnum} \\
\myname \\
\myemail \\
February 2021 \\
\end{center}

\question{1}{NFA vs DFA Expressiveness}
\part{1}\\
Let construct an NFA, $M = (Q,\Sigma,\delta,q_0,F) $ where $Q = \{0, 1, ..., k\}$. Let $\delta(0,b) = 0, \delta(0,1) = \{0,a\} = \{0,1\}$ and $\delta(i - 1, a) = i$, for $2 \leq i \leq k$. Then set $q_0 = 0$ and $F = \{k\}$. We know that the machine will start at state 0 (starting state). When the machine locate an $a$ it wil guess that it is a kth character to the right and will move to state 1. When it reaches state k, it will only accept if there are exactly $k-1$ bits following the one that move from $b$ to $a$.

\question{4}{HackerRank Challenge}
Natthakan Euaumpon\\
@natthakaneuaump1

\end{document}

