% You should title the file with a .tex extension (hw1.tex, for example)
\documentclass[a4paper, 11pt]{article}

\usepackage{amsmath}
\usepackage{amssymb}
\usepackage{fancyhdr}

\usepackage[margin=1in]{geometry}

\newcommand{\question}[2] {\vspace{.25in} \hrule\vspace{0.5em}
\noindent{\bf #1: #2} \vspace{0.5em}
\hrule \vspace{.10in}}
\renewcommand{\part}[1] {\vspace{.10in} {\bf (#1)}}

\newcommand{\myname}{Natthakan Euaumpon!}
\newcommand{\myemail}{natthakaneuaumpon@gmail.com}
\newcommand{\myhwnum}{19}

\setlength{\parindent}{0pt}
\setlength{\parskip}{5pt plus 1pt}
 
\pagestyle{fancyplain}
\lhead{\fancyplain{}{\textbf{HW\myhwnum}}}      % Note the different brackets!
\rhead{\fancyplain{}{\myname\\ \myemail}}
\chead{\fancyplain{}{ICCS310 }}

\begin{document}

\medskip                        % Skip a "medium" amount of space
                                % (latex determines what medium is)
                                % Also try: \bigskip, \littleskip

\thispagestyle{plain}
\begin{center}                  % Center the following lines
{\Large ICCS310: Assignment \myhwnum} \\
\myname \\
\myemail \\
March 2021 \\
\end{center}

\question{1}{Prove that $k-Coloring \leq_p (k+1)-Coloring$}
We know that k-Coloring is NP-Complete\\
Let the input to $(k+1)$Coloring be graph $G$. So now $G$ is $(k+1)$ colored we want to reduce $G$ to $k-Coloring$ by construct a new graph called $G^{'}$. We can done this by removing a node and every edges connected to that node. Hence $(k+1)$Coloring is NP-Complete.

\question{2}{Prove that $5-Coloring \leq_p 4-Coloring$}
We know that $4-Coloring$ is NP-Complete.\\
We can reduce 5-Coloring to 4-Coloring by mapping graph $G$ into a new graph $G^{'}$. Where $G \in 4-Coloring$ iff $G^{'} \in 5-Coloring$. This can be done by adding a new node y and connect them to each node in $G^{'}$. If $G$ is 4 colorable then $G^{'}$ can be 5 colored exactly as $G$ with a node y being the node that color with an additional color. Thus it is 4-Coloring and 4-Coloring is NP-Complete.

\question{3}{Argue that $P \subseteq coNP$, and hence $P \subseteq NP \cap coNP$}
WLOG let $x \in P$. We claim that $x \subseteq NP, \forall x \in P$. Then $x$ is use as a verifier in NP. We also claim that $x \subseteq coNP, \forall x \in P$. Then $x$ is use as a verifier in coNP as well. Therfore $P \subseteq coNP$, and hence $P \subseteq NP \cap coNP$.

\end{document}

